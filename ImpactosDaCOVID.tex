\documentclass[conference]{IEEEtran}
\IEEEoverridecommandlockouts
% The preceding line is only needed to identify funding in the first footnote. If that is unneeded, please comment it out.
\usepackage{cite}
\usepackage{amsmath,amssymb,amsfonts}
\usepackage{algorithmic}
\usepackage{graphicx}
\usepackage{textcomp}
\usepackage{xcolor}
\def\BibTeX{{\rm B\kern-.05em{\sc i\kern-.025em b}\kern-.08em
    T\kern-.1667em\lower.7ex\hbox{E}\kern-.125emX}}
\begin{document}

\title{Impactos da COVID-19 \\ no turismo brasiliense\\
{\footnotesize \textsuperscript{*}Note: Sub-titles are not captured in Xplore and
should not be used}
%\thanks{Identify applicable funding agency here. If none, delete this.}
}

\author{\IEEEauthorblockN{Eduardo Ofuji}
\IEEEauthorblockA{\textit{Engenharia da Computação} \\
\textit{IESB}\\
Brasília, Brasil \\
eduardo.ofuji@iesb.edu.br}
\and
\IEEEauthorblockN{Marco Paulo Nocello}
\IEEEauthorblockA{\textit{Ciências da Computação} \\
\textit{IESB}\\
Brasília, Brasil \\
marco.nocello@iesb.edu.br}
\and
\IEEEauthorblockN{Pedro Faleiros}
\IEEEauthorblockA{\textit{Ciências da Computação} \\
\textit{IESB}\\
Brasília, Brasil \\
pedro.faleiros@iesb.edu.br}
\and
\IEEEauthorblockN{Raymundo Guimarães}
\IEEEauthorblockA{\textit{Engenharia da Computação} \\
\textit{IESB}\\
Brasília, Brasil \\
raymundo.guimaraes@iesb.edu.br}
}

\maketitle

\section{Resumo}
O mercado turístico sofreu muito durante esses dois anos de pandemia. Quando o mundo todo deve que "congelar" para impedir a disseminação da COVID-19, várias empresas, cidades, comércio, entre outros que tinham seu sustento voltado para esse mercado tiveram que sobreviver ou até mesmo fechar suas portas, porém com a volta da "normalidade" a criação de um app que mostre a localização dos mesmos é uma forma muito eficiente para ajudar à alavancar esse mercado. Neste artigo será mostrado como a pandemia afetou os números de alguns lugares que necessitava de uma circulção de pessoas constante para se manter, e como uma aplicação móvel pode ajudar nesse problema. O objetivo é fazer o aplicativo rodar em smartphones com sistema Android, assim criando uma plataforma com um manuseio fácil e rápido para achar exatamente o que procura, assim criando um sistema de recomendação baseado no perfil do usuário. 

\begin{IEEEkeywords}
turismo, sistema de recomendação, Brasil, Brasília
\end{IEEEkeywords}

\section{Introdução}
O turismo é um dos pilares da economia de um país. Com uma pandemia que durou dois anos com inúmeras restrições pode-se dizer que esse ramo sofreu muito. Com a volta da "normalidade" oportunidades de ajudar a elevar o turismo surgem, e isso se soma mais ainda nos dias de hoje, que temos informações disponíveis facilmente em nossas mãos.

Muitas pessoas foram extremamentes preojudicadas por conta da quarentena que se estabeleceu ao redor do mundo, medida que era necessária para  combater a pandemia que surgiu. Em 2019 o Brasil registrou um movimento de 238,6 bilhões de reais, um aumento de 2,2 por cento em relação a 2017, último melhor ano registrado\cite{b1}. Porém com o início da pandemia em 2020, o setor registrou uma queda de 36 por cento, só havendo uma retomada positiva em 2021[2].Tendo em vista que o setor de turismo é um dos que mais movimenta dinheiro, e com as medidas de prevenção contra a COVID sendo melhor estabelecidas, em alguns países, esse ramo vai voltar a crescer e gerar mais receitas para locais que se beneficiam da alta circulção de pessoas.

Para ajudar esse crescimento, tanto para as empresas, quanto para quem procura locais de seu interesse, o desenvolvimento de um aplicativo que mostre exatamente o que procura é uma maneira viável e eficiente para a volta desse comércio. E também mostra informações necessárias para pessoas com necessidades, seja elas físicas ou alimentares, assim trazendo o máximo de dados possíveis para todo tipo de usuário\cite{b3}.



\section{Contexto}
Por meio de pesquisas, fica evidente que a porcentagem do turismo no brasil teve uma queda significante nos anos de 2020-2021. No começo do ano de 2022 a volta da "normalidade" que vivíamos está cada vez mais próxima, assim o setor de turismo aos poucos tem voltado a crescer novamente e com isso um grande demanda no setor de turismo.

\subsection{Problema}
No período da pandemia muitos locais que dependiam do turismo foram impedidas de continuar com suas atividades, alguns deles chegando ao ponto de fecharem suas portas. O que fazer para ajudar aqueles que conseguiram se manter durante esses dois anos e incentivar o turismo na capital do Brasil?

\section{Objetivo Geral}
Com essa ideia em mente, uma maneira de impulsionar o turismo brasiliense se tornou uma meta, e nos dias atuais existem várias maneiras que podem ajudar nesse objetivo sendo também uma maneira rápida e eficiente, dessa forma fazendo o setor de turismo crescer mais do que já tem sido mostrado nos últimos dados coletados\cite{b2}. 

\section{Obejtivos Específicos}
\begin{itemize}
    \item Criar uma paltaforma através de um aplicativo de dispositivo móvel que mostre as localidades de brasília;
    \item Implementar um sistema de recomendação para melhor experiência do usuário;
    \item Conseguir dados concretos de empresas no ramo de turismo.
\end{itemize}


\section{Referencial teórico}
Ao pesquisar em diversos artigos, foi encontrado um muito interessante o "App for Inclusive Tourism"\cite{b3}, que sua base de estudo consistia em fazer uma aplicação móvel que mostrasse locais que disponibilizavam de uma acessibilidade maior para pessoas com necessidades especiais, porém como é citado no próprio artigo o foco do app era deixar o turismo mais inclusivo. Tendo essa pesquisa como base e ampliando a ideia da mesma, que está sendo o foco dessa pesquisa. E na parte do sistema de recomendação foi utilizado uma filtragem colaborativa\cite{b4} para melhorar os locais recomendados.

\section{Trabalhos correlatos}
Para a pesquisa sobre o impacto da COVID-19 no setor de turismo, foram utilizados dados antes de começar a pandemia no brasil\cite{b1} e depois o quanto havia caído em 2020\cite{b2}, o ano que foi o grande ápice da doença, comparando os dois fica claro como essa situação global afetou muito o mercado turístico no Brasil, fazendo muitas pessoas do ramo fecharem suas portas diante da grande crise que ocorreu. Com a ideia de fazer uma aplicação móvel para ajudar o turismo muitos artigos foram pesquisados para se ter uma base, o \cite{b3} fala de um aplicativo de turismo inclusivo, já para o sistema de recomendação foi utilizado uma filtragem colaborativa \cite{b4} para otimizar ao máximo o sistema através de outros usuários.

\section{Metodologia}
As bibliotecas utilizadas foram "random" e "names" do python para gerar dois datasets. Sendo um deles com 10 mil nomes de lugares gerados aleatoriamente e cada um tendo seu próprio ID, o outro dataset com ID's de usuários seguidos com o ID de um local e uma nota de avaliação, também sendo aleatorios. Foi considerado que cada usuário tem no mínimo 3 e no máximo 100 avaliações. Colocou-se 1000 usuários avaliando aleatoriamente um dos 10 mil lugares com uma nota, também sendo aleatória, de 1 a 5. Para fazer a recomendação foi utilizado o algoritmo KNN e para isso foi utilizado "Nearest Neighbors" da biblioteca Scikit-Learn do Python. Utilizou-se a biblioteca "pandas" para modelar os dados dos datasets e criar um dataframe, esse mesmo dataframe foi utilizado no "Nearest Neighbors", usando a métrica do cosseno e o n\_neighbors igual a 4. Pegando então os 3 vizinhos mais próximos, pois com n\_neighbors igual a 4 ele irá retornar uma lista com os 4 vizinhos mais próximos, mas o primeiro elemento será ele mesmo.

\section{Resultados obtidos}
Ao visitar empresas do bairro Sudoeste foi obtido uma queda média de 22\% no faturamento, após o período da pandemia, infelizmente muitas empresas tiveram suas portas fechadas e não tiveram a mesma sorte daquelas que conseguiram permanecer. Por meio do Google Cloud foi implementado um mapa na aplicação móvel para visualização do mapa de brasília para os usuários.


Ao criar a plataforma é esperado que tenha todas informações possíveis dentro dela, para assim atrair uma grande variedade pessoas. E para satisfazer seus interesses a aplicação de um sistema de recomendação é uma ótima maneira para mostrar justamente o que a pessoa procura, assim mostrando no mapa as localidades de sua preferencia. Com a colaboração das empresas em compartilhar as informações para uma melhor base teórica e pesquisa.

\section{Conclusão}
Nesse artigo foi apresentado um trabalho sobre desenvolvimento de uma aplicação móvel em Android com o objetivo de aumentar o turismo em Brasília, mostrando restaurantes, pontos turísticos entre outros por meio da aplicação de um mapa pela Google Cloud, e para facilitar a experiência dos usuários foi desenvolvido um sistema de recomendação, por meio de filtragem colaborativa com as avaliações dos locais, assim mostrando possíveis locais de interesse para o usuário. Assim tentando contribuir ao máximo o mercado turístico brasiliense após 2 anos de pandemia, por conta da COVID-19, um setor que foi muito afetado ao redor do mundo e a criação de um app é uma das formas mais viáveis atualmente para ajudar nesse quesito.

Por mais que o objetivo do tarbalho tenha sido somente os locais de Brasília, fica como um trabalho futuro a expansão de território assim ajudando muitas localidades que dependem do setor de turismo para sua sobrevivência, muitos locais ao redor do Brasil tem sua economia baseada na ida e vinda de turistas em várias épocas do ano, então uma ampliação desse escopo seria um grande passo. 

\begin{thebibliography}{00}
\bibitem{b1} https://www.gov.br/turismo/pt-br/assuntos/noticias/turismo-movimentou-r-2386-bilhoes-no-brasil-em-2019-aumento-de-2c2
\bibitem{b2} https://www.gov.br/pt-br/noticias/viagens-e-turismo/2021/12/turismo-tem-retomada-em-2021-e-espera-aumento-na-geracao-de-empregos-no-setor
\bibitem{b3} Serra, R., Metrolho, J., and Ribeiro, F. (2019). App for Inclusive Tourism: A case study. 2019 14th Iberian Conference on Information Systems and Technologies (CISTI)
\bibitem{b4}PINHEIRO, Matheus Franca; DA SILVA, Nádia Félix Felipe. Sistemas de Recomendação com Filtros Colaborativos: um Estudo Comparativo. In: ESCOLA REGIONAL DE INFORMÁTICA DE GOIÁS (ERI-GO), 7. , 2019, Goiânia. Anais [...]. Porto Alegre: Sociedade Brasileira de Computação, 2019 . p. 155-168.
\end{thebibliography}

\end{document}
